\section{Conclusion}

\begin{frame}{The paradox of the ancient modernity}
Quantum Computing seems to live it's early industrial time, but has old and solid basements
\begin{itemize}
    \item Quantum Physics is amost 100 years old
    \item the theory behind Quantum Computing is more than 30 years old
    \item but "actual" QPUs exist for less than 10 years
\end{itemize}
We should keep in mind that 
\begin{itemize}
    \item Quantum Computing is at the crossroad of recent technologies with an old theory behind the curtains
    \item even theory can be updated and enhanced
\end{itemize}
\end{frame}

\begin{frame}{Several technological axes}
Quantum Computing has several technological axes
\begin{itemize}
    \item different paradigms : QUANTUM GATES, annealers, analogic QPU, photonics
    \item different "qubits" : photons, supraconducting loops, cat qubits, ions trapped, neutral atoms...
    \item different contraints : cryogeny, lasers, maanagement of magnetic fields in the machine room
\end{itemize}
In this context
\begin{itemize}
    \item Which technologies will scale up to thousands of qubits?
    \item energy management is critical, some solution are even more expensive than HPC
\end{itemize}
Today, it's hard to tell which technologies or paradigms will survive beyond 10 years.
\end{frame}

\begin{frame}{Controversies}
Quantum Computing was almost ignored until the "rise" of Shor's algorithm. It holds many controversies
\begin{itemize}
    \item can QC brings a \textit{quantum advantage} or will HPCalways perform better than QC?
    \item are QC algorithms only "toys for mathematicians" or will they have concrete usages? Grover's algorithm is
    more specifically concerned.
    \item is "quantum gates" paradigm reliable in a long term perspective.
    \item are analogic QPU's capabilities structurally mimited?
    \item will QC replace HPC?
    \item Will QC change HPC? (as the \textit{quantum inspired} algorithm did)
\end{itemize}
\end{frame}

\begin{frame}{Lots of forthcoming changes}
As Quantum Computing evolves, many interesting aspects remain
\begin{itemize}
    \item at the hardware level (with lots of quantum physics))
    \item at the algorithmical level (with lots of mathematics)
    \item at the system software level as HPC/QC integration will require to modify HPC environments
    \begin{itemize}
        \item via impacts on users environments and programming environments
        \item via strong impact on scheduling policies and mechanisms
        \item via new services and compute methods provided to the end users
    \end{itemize}
    \item at the middleware level : almost nothing exists... most is still to be done
\end{itemize}
\end{frame}
